% ************************** Thesis Abstract *****************************
% Use `abstract' as an option in the document class to print only the titlepage and the abstract.
\begin{abstract}
	The graph isomorphism problem (GI) has been proved to be solvable in quasi-polynomial time. This theoretical breakthrough does not necessarily describe the complexity of fast GI solvers.
	Following this discovery, a call was made to produce instances of graphs which execute slowly on these fast GI solving programs, namely Traces. The complexity of such solvers is benchmarked using numerous families of graphs and the construction of such difficult graphs is non-trival. In searching for a difficult family of graphs, there may be interesting results which bring light to the exploration of GI.
	\par
	We describe and implement a new variety of graphs using concepts in random 3-XOR-formulas on the threshold of satisfiability and multipedes. The rigid construction resists vertex-colour-refinement and automorphism factoring utilised in the backtracking search of Traces resulting in slow execution times. 
	\par
	A program for finding and executing such constructions is provided. In addition to an evaluation consisting of reproducing benchmark tests and comparing execution times. Furthermore, an analysis is provided on searching for these structures. Results showed that for many instances of our constructions, execution times were slow for graphs of equal node size in benchmark tests. Our construction consistently proved slow on Traces on a range of instances. Developing on previous work, graphs were small enough to be executed on Traces. Multiple packages were generated, which ranged in time complexity. Notably, those closest to the threshold of satisfiability showed the greatest execution time.
\end{abstract}
